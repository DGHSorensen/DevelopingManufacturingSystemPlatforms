%!TEX root = ..\dissertation.tex
\chapter*{Abstract}\addcontentsline{toc}{chapter}{Abstract}\label{Abstract}
%Introduction
Platforms have long since proven their worth for managing the ever-increasing variety of products, a trend experienced across the globe.
As collections of standardised assets, forming a structure from which a stream of derivative systems can be developed, platforms dictate which features of a product or system can be changed to achieve the desired variety and function.
For manufacturers, manufacturing system platforms are a way to achieve manufacturing systems capable of changing according their needs, accelerating new product introduction and development of new manufacturing systems.
Developing and implementing these manufacturing system platforms remains a challenging task for manufacturers, and a relatively immature field of research.

%Research Approach
To address these issues with manufacturing system platform development, this \PhD{} project employs a framework for design science in information systems research.
It combines design science and behavioural science, taking both a reactive and proactive approach to development of new and application of existing vocabulary, classifications, models, methods, and instantiations.
Inspiration is taken from product development, software architecture, and system engineering, using concepts from all three to grow the knowledge base on manufacturing system platforms, applying existing and new concepts in an industrial context.

%Contributions
The contributions of this research are documented in six appended papers, summarised in the thesis.
In a multi-case study, an iterative approach---employing concepts from software architecture and systems engineering---was used to guide the platform development process and vocabulary.
Several challenges appeared, highlighting issues to be addressed.
This lead to the development of a classification scheme for production processes and a summation of challenges related to manufacturing system platform development, based on an evolving case study carried out over three years.
This case study made the need for tools and objectivity clear, thus a classification coding scheme was developed, capturing key characteristics of manufacturing systems.
A method for brownfield platform development involving identification of potential platforms based on existing systems was proposed, and the classification coding scheme was demonstrated for this purpose.

%Conclusion

\vfill\noindent\textbf{Keywords:}\ \emph{Manufacturing;\ Platforms;\ Classification;\ Reconfigurable;\ Modularity;\ Reuse}