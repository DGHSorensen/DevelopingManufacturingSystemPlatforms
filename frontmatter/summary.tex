%!TEX root = ..\dissertation.tex
\chapter*{Dansk Resumé}\label{DKresume}\addcontentsline{toc}{chapter}{Dansk Resumé}
\begin{otherlanguage}{danish}
%Introduktion
Platforme har forlængst bevist deres værd når det kommer til at håndtere den stadigt stigende mængde af produktvarianter, en tendens der mærkes over hele kloden.
Som samlinger af standardiserede enheder, der danner en struktur hvorfra afledte systemer kan udvikles, dikterer platforme hvilke funktioner i et produkt eller system der kan ændres på for at opnå den ønskede variation og funktion.
For producenter er platforme en måde at opnå produktionssystemer, der kan ændres efter behov, accelerere introduktion af nye produkter, og udvikling af nye produktionssystemer.
Udvikling af disse produktionssystemplatforme er stadig en udfordring for producenter og et relativt umodent forskningsområde.

For at addressere disse problemer med udvikling af produktionssystemplatforme anvender dette \PhD{}-projekt et ``framework for design science in information systems research''.
Det kombinere designvidenskab og adfærdsvidenskab og tager derved en reaktiv og proaktiv tilgang til udvikling af nye, og anvendelse af eksisterende, klassifikationer, modeller, metoder, og instantieringer.
Inspiration hentes fra produktudvikling, softwarearkitektur, og system engineering, hvor koncepter fra alle tre bruges til at udvide vidensbasen om udvikling af produktionssystemplatforme ved at anvende eksisterende og nye koncepter i en industriel kontekst.

Forskningsbidragene er dokumenteret i seks vedlagte artikler og sammenfattet i denne afhandling.
I et multi case study blev en iterativ tilgang anvendt til at guide platformsudvikling ved brug af koncepter fra softwarearktitektur og system engineering.
Undervejs dukkede adskillige udfordringer op, og fremhævede problemer der måtte addresseres.
Dette førte til udvikling af et klassificeringssystem for produktionsprocesser, og en opsummering af udfordringer relateret til udvikling af produktionssystemplatforme baseret på et case study udført over tre år.
Dette case study tydeliggjorde behovet for objektive værktøjer, hvorfor et klassificeringskodningssystem, som beskriver nøgleegenskaber for produktionssystemer, blev udviklet.
En metode til brownfield-udvikling af platforme, der involverede identifikation af potentielle platforme baseret på eksisterende systemer, blev dernæst foreslået og klassificeringskodningssystemet blev demonstreret til dette formål.

\vfill\noindent\textbf{Nøgleord:}\ \emph{Produktion;\ Platforme;\ Klassificering;\ Rekonfigurering;\ Modularitet;\ Genbrug}
\end{otherlanguage}
