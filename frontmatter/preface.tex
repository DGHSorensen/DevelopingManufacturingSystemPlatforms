%!TEX root = ..\dissertation.tex
\chapter*{Preface}\label{acknow}\addcontentsline{toc}{chapter}{Preface}
If you've ever read the preface and acknowledgements section of a \PhD{} thesis before, you already know what's coming in this one.
They always talk about the amount of work being put into the thesis, the personal achievement it is to finish a work such as this, the many hours spent alone doing research, writing papers and finally the thesis.
Often, they will also talk about the author's motivation for starting this gruelling process, because what actually does motivate a person to go through this voluntarily?
We'll get back to that in a second.

The preface, acknowledgements or equivalent will \emph{always} talk about and thank a lot of people for their direct or indirect help with the thesis, the project and the papers.
In many ways, this preface will be no different.
But this is definitely a section I have been looking forward to writing.
I have been looking forward to putting the finishing touches on my thesis, of course, and thanking all the people who helped me along the way.
Credit where credit's due.

Having spent countless hours pouring over minor details just in the layout and setup of the thesis template, it has been a joy seeing the pages fill---sometimes slowly, sometimes fast---and everything just working.
But part of it, is also that I consider this one section of my thesis to be sort of a free space; a place where I can just write (almost) what comes to mind.
And truth be told, that is incredibly refreshing.
If you've read or written academic papers or theses before, you'll know that the language used in these kinds of things is something quite its own.
In this free space, I'm slightly less bound by scientific customs, terms and formality.
I can speak more directly to you.

Let's get back to the bit about motivation.
I started at Aalborg University in 2011, choosing mechanical engineering because it sounded right up my alley, and Aalborg University seemed like a perfect fit.
Lots of work in groups, projects focused on solving actual problems and applying actual solutions.
It wasn't easy by any means, but I liked it a lot---the subjects, the people and the university.
I didn't really find \emph{my} area of interest until the second semester of my Master's degree in 2015 where I had the first project on platforms.
It just clicked.
This was something I was good at, something I understood, something that interested me, and I was not about to let it go.
So for the following two last semesters, I kept working with platforms.
Building up more knowledge, coming up with new ideas for things to work on in relation to platforms.
By the end of it, I was worn out---a Master's thesis is not easy to write either.
Truth be told, the last thing I was considering at that point in time was doing a \PhD{}.
I decided I was done with the university and had to get out, at least for a little bit.
But then I was asked whether I wanted to do a \PhD{}, and after carefully considering it for a few months as I defended my Master's thesis and watched my friends do the same, I decided this was something I couldn't turn down.

And I do not regret it in the least.
Of course it didn't turn out exactly the way I'd imagined it, I don't think it ever really does, but it has definitely been an experience I wouldn't have been without.
I've met so many amazing people along the way, travelled to new places, learned so much and I've become a better person for it. 

So if you come across this thesis---whether it is in your search for knowledge, out of obligation because someone handed you a copy, if you find it buried on some shelf somewhere, or simply blindly grabbing \emph{something} to read---I hope you find it useful, interesting or at the very least, a way to kill a few hours.
Because all things considered and despite the bad times, I have had a blast working on this thesis and I \emph{am} proud of my work, this thesis being one of my bigger personal achievements so far.
And with that, before we head into the actual thesis, I have a few people I want to thank.

%!TEX root = ..\dissertation.tex
\section*{Acknowledgements}\label{ack}\addcontentsline{toc}{section}{Acknowledgements}

Firstly, my supervisors \AProf{} Thomas D. Brunø and \AProf{} Kjeld Nielsen.
You believed in me enough to give me this opportunity, and for that I am forever grateful.
The support, guidance, and inspiration you have given me over the last three have been invaluable.
And to my colleagues in the Mass Customization research group, I couldn't have dreamed of a better environment to work these past three years. 

My gratitude  goes out to \Prof{} Hoda A.\ ElMaraghy and \Prof{} Waguih ElMaraghy.
Thank you for welcoming me at the Intelligent Manufacturing System Centre at University of Windsor and sharing your knowledge with me.
It has been my honour to work with you, and your insight contributed significantly to the progress of my research.
In the same breath, I would like to thank the researchers at the Intelligent Manufacturing Systems Lab.
You helped make my stay enjoyable.

I would also like to thank Bjørn Langeland, chief engineer, for taking part in and helping me conduct this research with the industry.
We've had many long discussions throughout the years this research has taken place.
My acknowledgement also goes out to the Manufacturing Academy of Denmark (MADE) for giving me the opportunity to conduct this research.

Finally, I owe my family and friends more than words can express.
I would not have made it through this process without your support and help. 
A special mention and my heartfelt gratitude to Sofie Bech and Nikolai Øllegaard.
Thank you for being there for me through all of this.
For sharing my joy when things went right, for pulling me back up when I was down, and distracting me when I needed it.
I know I haven't always been easy to deal with, but you were always there for me.
Thank you for all the good times, and for everything that's come.
\begin{flushright}
\docAuthor%

Aalborg University, August 31st, 2019
\end{flushright}

%!TEX root = ..\dissertation.tex
\section*{Reader's Guide}\label{rGuide}\addcontentsline{toc}{section}{Reader's Guide}
This \PhD{} thesis takes the form of an extended summary covering a collection of papers structured into \cref{chp:Introduction,chp:Approach,chp:prodPltfDev,chp:Conclusions}.
References follow the author-year format, and are set in regular brackets, like so \parencite{SorensenMCPC2017}.
In the electronic version of the document, the entire citation acts as a hyperlink to the bibliography on \cpageref{bibliography}.
The bibliography itself contains the page number of each page a reference has been used.
Appended papers each have their own bibliography.
An index specifying several terms used in this thesis can be found on \cpageref{main}.

Internal cross-references between sections, pages, figures, and tables also act as hyperlinks to their corresponding parts of the thesis.
Figures and tables are numbered according to section in which they appear.

This document was written in \LaTeX~using SublimeText 3, typeset with Garamond and Arial, compiled using Lua\TeX.
A public repository of the source files is available at GitHub\footnote{\href{https://github.com/Firebrazer/DevelopingManufacturingSystemPlatforms}{github.com/Firebrazer/DevelopingManufacturingSystemPlatforms}}.
It complies with the guidelines set by Aalborg University, and follows a template developed by the author.
All figures were created in Microsoft Visio.
Appended papers follow templates for their corresponding journal or proceedings, but have been cropped and scaled to fit the pages of this thesis.