%!TEX root = ..\..\dissertation.tex
\section{Supporting Independent Development of Platforms}\label{sec:supPltfDev}
In summation, the rising demand for new products, new technologies, and options for personalisation all play a part in increasing the variety manufacturers must be able to handle in order to remain competitive.
Product platforms have successfully been employed to manage this variety in products, but dealing with variety and frequent product changes is straining the capabilities of existing manufacturing systems.
New manufacturing paradigms and the coming fourth industrial revolution present manufacturers with new opportunities to manage variety and improve the capabilities of their manufacturing systems.
Manufacturing system platforms and changeable manufacturing in conjunction with product platforms are seemingly attractive choices to achieve the desired variety and capability of both product and production.

Present effort is a step towards arming manufacturing companies with the necessary background, methods, and tools to independently develop manufacturing system platforms.
Few tools and methods exist aimed explicitly at assisting manufacturers in developing manufacturing system platforms.
Through the studies, projects, and research gathered and presented in this thesis, strides have been made to address this, by gaining inspiration and applying concepts, methods, and tools from various other fields dealing with systems engineering in general.

\subsection{Thesis Structure}\label{sec:thesisStructure}
% What will be presented in this thesis?
This thesis is structured into four chapters and an appendix with six appended papers.
The core purpose and content of each of the four chapters is described below:
\begin{itemize}[font={\normalfont}\bfseries]
  \item[\ref{chp:Introduction}] \textbf{\nameref{chp:Introduction}:}
  Sets the stage for the thesis, introducing its primary subject and context.
  Defines a number of important terms and provides state-of-the-art for manufacturing system platforms, linking it to other relevant research areas.
  \item[\ref{chp:Approach}] \textbf{\nameref{chp:Approach}:}
  Outlines the research approach, covering methodological background and introduces the industrial case.
  Lists research objective, questions and sub-questions addressed through the \PhD{} project.
  \item[\ref{chp:prodPltfDev}] \textbf{\nameref{chp:prodPltfDev}:}
  Describes the contributions made within the research field through the \PhD{} project, based off the six appended papers.
  Provides an extended abstract for each paper and summarises the implications of the contribution.
  \item[\ref{chp:Conclusions}] \textbf{\nameref{chp:Conclusions}:}
  Sums up the main findings of the \PhD{} project and addresses research objectives and questions.
  Discusses findings and applications, and outlines future research.
\end{itemize}