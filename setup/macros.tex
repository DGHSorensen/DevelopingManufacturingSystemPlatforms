%!TEX root = ..\dissertation.tex
%%%%%%%%%%%%%%%%%%%%%%%%%%%%%%%%%%%%%%%%%%%%%%%%
% Math
%%%%%%%%%%%%%%%%%%%%%%%%%%%%%%%%%%%%%%%%%%%%%%%%
%Brackets - Should only be used in math mode
\newcommand{\lrp}[1]{\left(#1\right)} %Paranthesis
\newcommand{\lrb}[1]{\left[#1\right]} %Square brackets
\newcommand{\lrB}[1]{\left\lbrace#1\right\rbrace} %Curly brackets
\newcommand{\lrf}[1]{\left\lfloor#1\right\rfloor} %Floor brackets
\newcommand{\lra}[1]{\left\langle#1\right\rangle} %Angle brackets

%Misc Symbols
\newcommand{\diam}{\ensuremath{\diameter}} %Short form of diameter symbol
\newcommand{\al}{\ensuremath{\alpha}} %Short form of alpha symbol
\newcommand{\bt}{\ensuremath{\beta}} %Short form of beta symbol
\newcommand{\de}{\ensuremath{\delta}} %Short form of delta symbol
\newcommand{\ve}{\ensuremath{\varepsilon}} %Short form of epsilon symbol
\newcommand{\ga}{\ensuremath{\gamma}} %Short form of gamma symbol

\newcommand{\md}{\ensuremath{\,\mathrm{d}}} %Math d for derivation

% Swap the definition of \abs* and \norm*, so that \abs
% and \norm resizes the size of the brackets, and the 
% starred version does not.
\makeatletter
  \let\oldabs\abs%
  \def\abs{\@ifstar{\oldabs}{\oldabs*}}

  \let\oldnorm\norm%
  \def\norm{\@ifstar{\oldnorm}{\oldnorm*}}
\makeatother

% New definition of square root
% Renames \sqrt as \oldsqrt
\let\oldsqrt\sqrt%
% Defines the new \sqrt in terms of the old one
\def\sqrt{\mathpalette\DHLhksqrt}

\def\DHLhksqrt#1#2{%
  \setbox0=\hbox{$#1\oldsqrt{#2\,}$}\dimen0=\ht0
  \advance\dimen0-0.2\ht0
  \setbox2=\hbox{\vrule height\ht0 depth -\dimen0}%
  {\box0\lower0.4pt\box2}}


%%%%%%%%%%%%%%%%%%%%%%%%%%%%%%%%%%%%%%%%%%%%%%%%
% SI Units in brackets
%%%%%%%%%%%%%%%%%%%%%%%%%%%%%%%%%%%%%%%%%%%%%%%%
%The siunitx package is used to typeset SI units in the main text. For equations tables and the like, the bracketed version is used
%It takes the exact same inputs as the siunitx commands
\newcommand{\bsi}[1]{\ensuremath{\left[\si[per-mode=symbol]{#1}\right]}} %Units in brackets
\newcommand{\bSI}[2]{\num{#1}\bsi{#2}} %Number in SI mode, units in brackets


%%%%%%%%%%%%%%%%%%%%%%%%%%%%%%%%%%%%%%%%%%%%%%%%
% Common Abbreviations
%%%%%%%%%%%%%%%%%%%%%%%%%%%%%%%%%%%%%%%%%%%%%%%%
\newcommand*{\etc}{\xperiodafter{etc}}
\newcommand*{\eg}{\xperiodafter{e.g}}
\newcommand*{\ie}{\xperiodafter{i.e}}
\newcommand*{\cf}{\xperiodafter{cf}}

\newcommand*{\BSc}{\xperiodafter{B.Sc}}
\newcommand*{\MSc}{\xperiodafter{M.Sc}}
\newcommand*{\PhD}{\xperiodafter{Ph.D}}
\newcommand*{\AProf}{\xperiodafter{Assoc.\@ Prof}}
\newcommand{\Prof}{\xperiodafter{Prof}}

\newcommand*{\ISOAD}{ISO/IEC/IEEE 42010:2011(E)}


%%%%%%%%%%%%%%%%%%%%%%%%%%%%%%%%%%%%%%%%%%%%%%%%
% Prevent indents after environments
%%%%%%%%%%%%%%%%%%%%%%%%%%%%%%%%%%%%%%%%%%%%%%%%
\makeatletter
% uncomment the following if you don't want \clubpenalty\@M ...
% \let\nearly@afterheading\@afterheading
% \patchcmd\nearly@afterheading
%   {\@M}% original temporary setting for \clubpenalty replaced by ...
%   {\@clubpenalty}% ... or whichever value you deem right
%   {}{}
% ... and use \nearly@afterheading instead of \@afterheading here:
  \newcommand*\NoIndentAfterEnv[1]{%
    \AfterEndEnvironment{#1}{\par\@afterindentfalse\@afterheading}}
\makeatother


%%%%%%%%%%%%%%%%%%%%%%%%%%%%%%%%%%%%%%%%%%%%%%%%
% List of Abbreviations environment
%%%%%%%%%%%%%%%%%%%%%%%%%%%%%%%%%%%%%%%%%%%%%%%%
\makeatletter
  \newcommand{\tocfill}{\cleaders\hbox{$\m@th \mkern\@dotsep mu . \mkern\@dotsep mu$}\hfill} %Dots spaced liked the TOC
\makeatother

\newcommand{\abbrlabel}[1]{\makebox[3cm][l]{\textbf{#1}\ \tocfill}}

\newenvironment{abbreviations}{\begin{list}{}{\renewcommand{\makelabel}{\abbrlabel}%
        \setlength{\labelwidth}{3cm}\setlength{\leftmargin}{\labelwidth+\labelsep}%
                                              \setlength{\itemsep}{0pt}}}{\end{list}}


%%%%%%%%%%%%%%%%%%%%%%%%%%%%%%%%%%%%%%%%%%%%%%%%
% Theorems
%%%%%%%%%%%%%%%%%%%%%%%%%%%%%%%%%%%%%%%%%%%%%%%%
% New example environment
\theoremheaderfont{\normalfont\bfseries}% Set example header font
\theorembodyfont{\normalfont}% Set example body font
\theoremstyle{break}% Set example theorem style

\def\theoremframecommand{{\color{aaugrey}\vrule width 5pt \hspace{5pt}}}% Makes a grey bar to the left 

\newshadedtheorem{exa}{Example}[section]% Set header and reset counter for every section

\newenvironment{example}[1]{% New environment
		\begin{exa}[#1]
}{%
		\end{exa}
}

% New definition environment
\makeatletter
\newtheoremstyle{deftheorem}% New theorem style based on nonumberbreak
    {\item[\rlap{\vbox{\hbox{\hskip\labelsep \theorem@headerfont% Defines default style
        ##1\theorem@separator}\hbox{\strut}}}]}%
    {\item[\rlap{\vbox{\hbox{\hskip\labelsep \theorem@headerfont% Defines style when optional input is added
        ##1\ ##3\theorem@separator:}\hbox{\strut}}}]}% Removed brackets and added colon
\makeatother

\theoremheaderfont{\normalfont\bfseries}% Set definition header font
\theorembodyfont{\normalfont}% Set definition body font
\theoremstyle{deftheorem}% Set definition theorem style
\def\theoremframecommand{{\color{aaugrey2}\vrule width 5pt \hspace{5pt}}}% Makes a grey bar to the left 
\newshadedtheorem{defin}{Definition of}% Set header
\newenvironment{definition}[1]{% New environment
    \begin{defin}[#1]
}{%
    \end{defin}
}
\NoIndentAfterEnv{definition}% Prevent any indents after definition environment

% New Research Question environment
\makeatletter
\newcounter{resqc}
\newtheoremstyle{rqtheorem}% New theorem style based on nonumberbreak
    {\item[\rlap{\vbox{\hbox{\hskip\labelsep \theorem@headerfont%
        ##1\ \stepcounter{resqc}\arabic{resqc}\theorem@separator:}\hbox{\strut}}}]}%
\makeatother

\theoremheaderfont{\normalfont\bfseries}% Set RQ header font
\theorembodyfont{\normalfont}% Set RQ body font
\theoremstyle{rqtheorem}% Set RQ theorem style
\def\theoremframecommand{{\color{aaudarkgrey}\vrule width 5pt \hspace{5pt}}}% Makes a dark grey bar to the left
\newshadedtheorem{resq}{Research Question}% Set header
\newenvironment{researchQ}{% New environment
    \begin{resq}
}{%
    \end{resq}
}
\NoIndentAfterEnv{researchQ}% Prevent any indents after RQ environment

%New Thesis Objective environment
\makeatletter
\newtheoremstyle{objtheorem}% New theorem style based on nonumberbreak
    {\item[\rlap{\vbox{\hbox{\hskip\labelsep \theorem@headerfont% Defines default style
        ##1\theorem@separator}\hbox{\strut}}}]}%
\makeatother

\theoremheaderfont{\normalfont\bfseries}% Set objective header font
\theorembodyfont{\normalfont}% Set objective body font
\theoremstyle{objtheorem}% Set objective theorem style
\def\theoremframecommand{{\color{aaublue}\vrule width 5pt \hspace{5pt}}}% Makes a grey bar to the left 
\newshadedtheorem{objec}{Research Objective}% Set header
\newenvironment{objective}{% New environment
    \begin{objec}
}{%
    \end{objec}
}
\NoIndentAfterEnv{objective}% Prevent any indents after objective environment

%%%%%%%%%%%%%%%%%%%%%%%%%%%%%%%%%%%%%%%%%%%%%%%%
% Bibliography
%%%%%%%%%%%%%%%%%%%%%%%%%%%%%%%%%%%%%%%%%%%%%%%%
%Possesive Cite
\DeclareNameFormat{labelname:poss}{% Based on labelname from biblatex.def
  \nameparts{#1}% Not needed if using Biblatex 3.4
  \ifcase\value{uniquename}%
    \usebibmacro{name:family}{\namepartfamily}{\namepartgiven}{\namepartprefix}{\namepartsuffix}%
  \or%
    \ifuseprefix%
      {\usebibmacro{name:given-family}{\namepartfamily}{\namepartgiveni}{\namepartprefix}{\namepartsuffixi}}
      {\usebibmacro{name:given-family}{\namepartfamily}{\namepartgiveni}{\namepartprefixi}{\namepartsuffixi}}%
  \or%
    \usebibmacro{name:first-last}{\namepartfamily}{\namepartgiven}{\namepartprefix}{\namepartsuffix}%
  \fi
  \usebibmacro{name:andothers}%
  \ifnumequal{\value{listcount}}{\value{liststop}}{'s}{}}
\DeclareFieldFormat{shorthand:poss}{%
  \ifnameundef{labelname}{#1's}{#1}}
\DeclareFieldFormat{citetitle:poss}{\mkbibemph{#1}'s}
\DeclareFieldFormat{label:poss}{#1's}
\newrobustcmd*{\posscitealias}{%
  \AtNextCite{%
    \DeclareNameAlias{labelname}{labelname:poss}%
    \DeclareFieldAlias{shorthand}{shorthand:poss}%
    \DeclareFieldAlias{citetitle}{citetitle:poss}%
    \DeclareFieldAlias{label}{label:poss}}}
\newrobustcmd*{\posscite}{%
  \posscitealias%
  \textcite}
\newrobustcmd*{\Posscite}{\bibsentence\posscite}
\newrobustcmd*{\posscites}{%
  \posscitealias%
  \textcites}